This section explores these aspects in greater detail to evaluate the feasibility of adopting ML-DSA in a live Ethereum environment.The simulation confirms that, conceptually, transitioning to post-quantum signatures such as ML-DSA is achievable without fundamentally altering Ethereum’s blockchain structure or consensus mechanism.

\subsection*{EVM Compatibility and Smart Contract Integration}
A significant technical consideration is compatibility with the Ethereum Virtual Machine (EVM). The current EVM is optimized for elliptic curve arithmetic, particularly for ECDSA-based signatures. By contrast, ML-DSA relies on polynomial operations and modular arithmetic over lattice structures. While not fundamentally incompatible, this would necessitate the addition of new precompiled contracts (as seen in EIPs\footnote{Ethereum Improvement Proposal, design documents providing information to the Ethereum community about new features, standards, or changes to the protocol, see \url{https://ethereum.org/en/eips/\#what-are-eips}} like EIP-197\cite{eip197} and EIP-2537\cite{eip2537} for cryptographic curves) and potentially even new opcodes or gas metering adjustments. Alternatively, integration through eWASM\cite{ewasm2019}, (a proposed upgrade to the Ethereum Virtual Machine (EVM)) may allow for more flexible cryptographic primitives in future versions of Ethereum.

\subsection*{Transaction Format and Network Propagation}
ML-DSA signatures are significantly larger than those produced by ECDSA. Larger signatures directly translate to increased bandwidth usage, slower transaction propagation, and higher gas fees. Moreover, block size and gas usage policies would need revision to prevent issues caused by inflated payloads. These risks could be mitigated with layered solutions such as compression or post-quantum aggregation, but those remain active areas of research.

\subsection*{Backward Compatibility and Governance Overhead}
Even though a progressive rollout strategy was simulated (via Phases 1 to 4), a full transition to post-quantum signatures would eventually require a hard fork to enforce the use of ML-DSA at the consensus level. This would entail governance coordination similar to past major forks like Byzantium\cite{eip609} or London\cite{eip3238}. During the transition period, dual-signature support must be managed securely, with consensus clients verifying both types of signatures. This requires canonical rules to avoid consensus failure between clients. Additionally, tooling and protocol-level auditability must be extended to accommodate post-quantum key formats and their serialization.

\subsection*{Key Management and Validator Operations}
The simulation assumes validators possess both ECDSA and ML-DSA key pairs. In practice, managing dual key infrastructures at scale poses operational and security risks. Existing tools (e.g., MetaMask, Ledger, or staking services) are deeply integrated with elliptic curve formats. Introducing lattice-based formats would require new standards for seed generation, serialization, and HD key derivation. Furthermore, on-chain staking and slashing conditions would need to recognize and enforce rules for both key types, increasing protocol complexity and the likelihood of misconfiguration during the transition.These operational challenges could increase the risk of validator misbehavior or misconfiguration, especially during the transition period, and would require extensive tooling updates and stakeholder education.\\

Although not without integration challenges, ML-DSA offers a pragmatic and forward-compatible path toward post-quantum security for Ethereum—provided that the ecosystem evolves to accommodate larger keys, new cryptographic primitives, and updated operational standards.
