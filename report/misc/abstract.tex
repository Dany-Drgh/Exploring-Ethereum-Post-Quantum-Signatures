The development of quantum computing poses a serious threat to classical cryptographic primitives, particularly those used in securing blockchain systems such as Ethereum. As these systems rely on digital signatures for transaction authentication and consensus, transitioning to post-quantum secure algorithms is essential to maintaining long-term integrity and trust.

This thesis investigates the feasibility, complexity, and efficiency of integrating post-quantum digital signature schemes into the Ethereum blockchain. Focusing on three NIST-standardized candidates, SLH-DSA (SPHINCS+), ML-DSA, and FALCON, it analyzes their cryptographic properties, performance characteristics, and compatibility with Ethereum's hybrid Proof-of-Stake architecture. ML-DSA was identified as the most suitable candidate, offering a favorable balance between security guarantees and computational efficiency.

A multi-phase transition strategy was proposed and implemented in a simulation-based environment to assess the impact of these schemes on gas cost, signature size, and consensus performance. The results indicate that Ethereum can incorporate post-quantum signatures without prohibitive performance penalties, especially when compared to post-quantum Proof-of-Work alternatives. The thesis also explores how lattice-based signatures may support privacy enhancements, particularly in combination with zero-knowledge proof systems.

These findings hope to contribute to the broader effort to prepare decentralized platforms for a post-quantum future, offering a path toward secure, scalable, and privacy-preserving blockchain ecosystems.