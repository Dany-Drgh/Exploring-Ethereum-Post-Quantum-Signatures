%Intro to hash functions first and then:SHA-256, KECCAK
\subsubsection*{Definition and Concept:}
A hash function $H(m), H:\{0,1\}^{*} \rightarrow {0,1}^{n}$ is a function that maps an input (message) of arbitrary length, to a fixed size output (digest or hash).
Hash functions have properties making them a fundamental building block in cryptography, and especially in blockchain technology:
\begin{itemize}
    \item \textbf{Pre-image resistance:} Given a hash value $h$, it is computationally impossible to find any message $m$ such that $H(m) = h$.
    %%
    \item \textbf{Second pre-image resistance:} Given a message $m$, it is computationally impossible to find another message $m'$ such that $H(m) = H(m')$.
    %%
    \item \textbf{Collision resistance:} It is computationally impossible to find two different messages $m$ and $m'$ such that $H(m) = H(m')$.
    %%
    \item \textbf{Deterministic:} For a given input, the output is always the same.
    %%
    \item \textbf{Efficient:} The computation of the hash value is fast. 
\end{itemize}

\subsubsection{SHA-256}
SHA-256 is a member of the SHA-2 family of hash functions \cite{nist_sha256} developed by NIST, that produces a 256-bit digest, and is widely used in proof-of-work blockchains and digital signatures as it shows a strong collision resistance, and is computationally efficient. 

\subsubsection*{Algorithm:}
\begin{enumerate}
    \item \textbf{Padding:} The message is padded to a length that is a multiple of 512 bits.
    %%
    \item \textbf{Initial Hash Values:} 8 Initial hash values are chosen based on the square roots of prime numbers.
    %%
    \item \textbf{Processing:} The message is processed in 512-bit blocks, and the hash value is updated after each block. 
    %%
    \item \textbf{Output:} The final hash value is the concatenation of the 8 32-bit hash values as a 256-bit digest.
\end{enumerate}

\begin{figure}[htp]
    \centering
    \includegraphics[width=0.8\textwidth]{images/Sha-256.png}
    \caption{SHA-256 Algorithm}    
\end{figure}

Although not used in Ethereum for block hashing, SHA-256 is used in its proof-of-work algorithm, Ethash\cite{EthereumEthash}, and in digital signatures.
%------------------------------------------------------------------------------%
\subsubsection{KECCAK}
KEECCAK\cite{keccak_sha3} is the family of hash functions that won the NIST SHA-3 competition, and is based on the sponge construction. It is used in Ethereum for block hashing, and in digital signatures and produces a 256-bit digest.

\begin{figure}[H]
    \centering
    \begin{tabular}{c|c|c}
        \textbf{Property} & \textbf{SHA-256} & \textbf{KECCAK} \\
        \hline
        Rounds & 64 & 24 \\
        \hline
        Performance & Faster on general hardware & More efficient on hardware-accelerated implementations. \\
        \hline
        Usage in Ethereum & Used in PoW mining & Used for block hashing, address generation and state hashing. \\
    \end{tabular}
    \caption{Comparison of SHA-256 and KECCAK hash functions}
    \label{fig:hash_comparison}
\end{figure}

The particular construction of the KECCAK hash function allows for variable output sizes through its multi-phase approach, in the absorbing phase, the message is XORed with a representation of the current state, and in the squeezing phase, the output is generated by reading the state. \\

\begin{figure}[htp]
    \centering
    \includegraphics[width=0.8\textwidth]{images/KECCAK.png}
    \caption{KECCAK Algorithm}
    \label{fig:keccak}
\end{figure}
