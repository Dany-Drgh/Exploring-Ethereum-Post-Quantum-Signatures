Hash-Based Cryptography in the context of digital signature \cite{merkle1988} is the approach that relies on the security of hash functions, provided by their mathematical properties such as preimage resistance, second-preimage resistance and collision resistance  (explained in \hyperref[sec:hash_functions]{section 2.2}), as opposed to relying on number theory (like DSA / ECDSA).\\

Hash-based signatures are considered to be resistant to quantum attacks \cite{bernstein2015hash}, and can further mitigate any threats by expanding output sizes from 256 to 512 bits, or even 1024 bits. Making them a suitable candidate for post-quantum cryptography, however this increase in output size might cause problems in terms of performance in the context of blockchain technology and Ethereum.\\