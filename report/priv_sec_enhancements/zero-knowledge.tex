Zero knowledge proofs\cite{Goldwasser1985ZK}, a cryptographic protocols that allow one party to prove to another that a statement is true without revealing any additional information beyond the validity of the statement itself; are a powerful tool already used in Ethereum to verify smart contract and transaction correctness through zk-SNARKs\cite{Parno2013Pinocchio}, zk-STARKS\cites{BenSasson2018Stark}, and zk-Rollups\cite{EthereumZKRollups}, and considering post-quantum cryptography will be an important aspect to keep these constructs secure against quantum threats. While independent from the signature layer, the inclusion of zero-knowledge proofs in Ethereum's protocol stack is crucial for enhancing privacy, scalability, and security.

Some currently used zero-knowledge proof systems like zk-SNARKs\cite{Parno2013Pinocchio} rely on elliptic curves, making them vulnerable to quantum attacks. A challenge posed by quantum resistant ZKPs, such as zk-STARK and lattice-base zk-SNARKs\cite{Chiesa2023PostQuantumSNARKs} is that they tend to have large proof sizes and induce a heavy computational overhead. They however are being studied and developed for practical use: 
\begin{itemize}
    \item zk-STARKs: quantum-secure (since they only rely on hash functions), transparent (no trusted setup), scalable and support large computations but come with large proof sizes and significant gas and bandwidth costs in blockchain environments like Ethereum.
    
    \item Lattice-based zk-SNARKs\cite{lyubashevsky2023latticesnark}: designed to offer compact, post-quantum-secure, and universal zero-knowledge proofs. Unlike STARKs, lattice-based SNARKs can maintain relatively shorter proofs and better integration with Ethereum's infrastructure, especially if Ethereum adopts ML-DSA, which is also based on lattice assumptions.
\end{itemize}

By selecting ML-DSA as the primary signature scheme, Ethereum would already be operating in a lattice-based cryptographic context, easing the transition to lattice-based zk-SNARKs, opening the door to developing cryptographically unified proof systems, reducing the complexity of maintaining hybrid cryptographic stacks and potentially optimizing performance through shared primitives (e.g., ring/module sampling, Gaussian noise, or polynomial arithmetic).

However, lattice-based ZKP systems are still in an early phase. Active research continues to improve their efficiency, trusted setup requirements, and compatibility with practical constraint systems. Recent proposals such as qSNARK\cite{benhamouda2022qsnark}, LigeroLWE\cite{albrecht2021ligerolwe}, and Module-LWE-based Groth-like constructions \cite{chase2022grothlattice} show promise in achieving practical zero-knowledge proofs over lattices, but they have yet to reach the same maturity and performance optimization as classical alternatives.

These developments suggest that future-proofing Ethereum's privacy and scalability stack will require cohesive advances in both post-quantum signature schemes and zero-knowledge proof constructions.