Post-quantum signature schemes, namely the ones based on lattice structures, offer new opportunities for privacy improvements in decentralized networks like Ethereum. While digital signatures are not inherently designed for privacy, some post-quantum primitives possess structural features that enable privacy-preserving protocols.

A key privacy concern, in Blockchains and namely in Ethereum is linkability\cite{Meiklejohn2016Fistful}\cite{Androulaki2013PrivacyBitcoin}, the ability of observers to link multiple transactions to the same user based on recurring digital signature patterns. This can be exploited by adversaries performing transaction graph analysis, enabling the de-anonymization of users even in pseudonymous settings. In ECDSA, the signature structure reuse can facilitate these analysis.

Lattice-based schemes like ML-DSA offer a mitigation this threat by incorporating intrinsic randomness in the signing process. Each signature, even when generated from the same key and message, is statistically independent due to the noise sampling from high-entropy distributions. This property significantly increases unlinkability, reducing the risk of identifying recurring actors in the network.

Additionally, the large key and randomness spaces inherent to lattice-based schemes make pre-image and correlation attacks computationally infeasible even for powerful adversaries. When deployed within Ethereum's account-based model, ML-DSA could serve as the basis for enhanced privacy-preserving constructs, supporting untraceable transaction patterns without requiring fundamental changes to the blockchain architecture.