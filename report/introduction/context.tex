%1. The growing role of blockchain and Ethereum
Blockchain technology has established itself as a foundational innovation in decentralized systems, with applications spanning finance, governance, and data security. Among the various blockchain platforms, Ethereum has distinguished itself by enabling smart contracts, self-executing programs that operate without 
intermediaries, powering decentralized finance (DeFi), non-fungible tokens (NFTs), and various solutions as such, and especially as  Ethereum transitions to a hybrid Proof of Work (PoW) and Proof of Stake (PoS) model, the long-term security of its cryptographic infrastructure is a critical issue, as of today, Ethereum relies on cryptographic primitives that are vulnerable to quantum attacks, such as the Elliptic Curve Digital Signature Algorithm (ECDSA) used for transaction signing. However, as the field of quantum computing develops, the cryptographic basis of these algorithms face the risk of being broken.\\

%2. The quantum threat
Quantum computers use principles of superposition and entanglement to perform computations that can not be done by classical machines. Namely, Shor's algorithm\cite{shor1994}, a quantum algorithm for integer factorization and discrete logarithm computation, puts the security of RSA and elliptic curve cryptography (ECC), including Ethereum’s ECDSA-based signatures at high risk.\\
Though quantum computers are still at the early stages of development, industry leaders such as Google \cite{Arute2019} and IBM \cite{IBMQuantum2023} have demonstrated significant progress, leading experts to believe that the threat of quantum computing to current cryptographic standards is going to be a real and practical issue in the foreseeable future.\\

% 3. Preparation for the quantum threat

In order to address this threat, the National Institute of standards and technology (NIST) have started a selection process within post-quantum cryptography algorithms to find secure and efficient alternatives to current schemes. In addition, the Ethereum foundation acknowledged the importance of post quantum cryptography and has made research and work on developing quantum-resistant schemes.\\

% 4. Motivation - ish
While post-quantum cryptography has been an active research field,  most studies focus on general transitions rather than the unique challenges posed by Ethereum's hybrid model. This thesis aims to bridge that gap by exploring the feasibility, complexity, and efficiency of integrating quantum-resistant signatures into Ethereum.\\