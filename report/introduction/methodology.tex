\subsubsection{Research Approach}

Both a theoretical approach to the topic and an experimental implementation and performance evaluation are proposed, in order to understand and evaluate the candidate algorithms and a possible implementation strategy. The theoretical part consists of a study of the proposed signature algorithms and their mathematical foundations, as well as their efficiency, and suitability for blockchain environments.

The experimental aspect consists of a proof-of-concept (PoC) implementation of how Ethereum would look like if selected post-quantum signature schemes were put in place. While the PoC will not be a full-scale Ethereum implementation, it will serve as a simplified model for evaluating the impact of such an integration integration. 

\subsubsection{Algorithm Selection}
In order to select the most suitable candidate, the leading candidate algorithms proposed by the latest NIST status report \cite{NISTPQC2022} will be evaluated based on the following criteria:
\begin{itemize}
    \item Security: The security of the algorithm against known attacks.
    \item Resistance to Quantum Attacks: The algorithm's resistance to quantum attacks.
    \item Signature size: The size of the signature generated by the algorithm.
    \item Key size: The size of the public and private keys generated by the algorithm.
    \item Computational efficiency; The time taken to generate a signature and verify it.
\end{itemize}

The proposed algorithms will also be evaluated based on its compatibility with the Ethereum blockchain and its ability to be integrated into the existing Ethereum infrastructure, although this consideration is also dependent on the implementation strategy. 
A single scheme meeting the above criteria, will be selected as the basis for the PoC implementation.

\subsubsection{Implementation Strategy}
The implementation strategy will be based on the selected post-quantum signature scheme, and will focus on the integration of the scheme into the Ethereum 2.0 blockchain. The implementation will be designed to be backward compatible with the existing model, and will also have to take the security of past transactions into consideration.
In addition to these concerns, the implementation will also measure the integration's impact on the Ethereum network's performance, and security. 

\subsubsection{Evaluation Methodology}

Evaluation of the system will consider security, performance, and scalability. For security evaluations, the system will be evaluated based on the security properties of the selected post-quantum signature scheme, as well as the system's resistance to quantum attacks, and compared to the current Ethereum blockchain as well as classic proof-of-work one for reference. For performance, time taken to generate a signature and verify it will be measured, as well as the overall throughput of the system. 


\subsubsection{Research Limitations}

Although this study aims at providing a thorough evaluation of the impact of post-quantum signature schemes on the Ethereum blockchain, some limitations are to be considered. The proof-of-concept (PoC) implementation will be conducted on a simplified Ethereum blockchain model, meaning that certain real-world factors such as network congestion, miner incentives, and full Ethereum client integration, will not be fully taken into account. In addition, the current state of the Ethereum blockchain is not optimized for some of the schemes, namely lattice-based schemes, thus impact on the exact impact of the integration on gas costs in a production environment may require further testing on Ethereum testnets. Another key limitation is the assumption that a gradual transition away from ECDSA verification is feasible, whereas in reality, the process may face adoption resistance form the community as well as unforeseen challenges. Despite these limitations the study aims to provide a comprehensive evaluation of the impact of post-quantum signature schemes on the Ethereum blockchain, and to provide insights into the feasibility of such an integration.