Recent and upcoming breakthroughs in the field of quantum computing present a considerable challenge to the security of modern cryptographic schemes, namely in blockchain technology. Ethereum\footnote{Link to the Ethereum website \url{https://ethereum.org/fr/}} as one of the leading platforms relies on primitives such as the Elliptic Curve Digital Signature Algorithm (ECDSA) for transaction verification and network security. However such primitive are vulnerable to quantum attacks such as Shor's algorithm \cite{shor1994} possibly rendering them obsolete in the near future. As Ethtereum transitions from proof-of-work to proof-of-stake, exploring post-quantum resistant solutions becomes a necessity to ensure the long-term security of the platform.This thesis investigates the feasibility, complexity, and efficiency of integrating quantum-resistant signature schemes into Ethereum's blockchain, addressing both theoretical and practical considerations. 

\textit{A proof-of-concept implementation is proposed along with this study,
 and available at \\ \url{https://gitlab.unige.ch/Dany.Al-Moghrabi/exploring-ethereum-post-quantum-signatures}}

