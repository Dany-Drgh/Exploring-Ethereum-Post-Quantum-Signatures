As quantum computing continues to evolve from theoretical to practical threat, ensuring the long-term security of blockchain systems has become a critical concern. In this context, the Ethereum blockchain,central to a wide array of decentralized applications, must adapt to maintain its robustness in a post-quantum world.\\

This thesis explored the feasibility, efficiency, and integration challenges of post-quantum digital signature schemes within Ethereum's hybrid Proof-of-Stake framework. After a thorough evaluation of the leading candidates standardized by NIST, SLH-DSA (SPHINCS+), ML-DSA, and FALCON, the analysis focused on their structural properties, implementation overhead, and compatibility with Ethereum's account-based model. ML-DSA appeared to be the better choice, showing promise for deployment due to its balance of security and performance. A simulation-based implementation strategy was proposed and assessed across five blockchain configurations, allowing for empirical comparisons under realistic conditions.\\

While post-quantum schemes provide quantum resilience, they also introduce significant trade-offs particularly in terms of signature size, gas cost, and computational overhead, although keeping the expected performance better than that of a Post-Quantum adapted proof-of-work blockchain, namely in terms of block-time. PoW systems on the other hand, would require less effort to adapt to quantum threats.\\ 
These aspects must be carefully balanced against the operational constraints of Ethereum's network.\\ 
In addition, the thesis explored how post-quantum cryptography could augment privacy features, particularly in conjunction with zero-knowledge proofs. Although these areas are still under active research, integrating post-quantum ZKPs may play a pivotal role in the evolution of private, scalable, and quantum-resistant smart contracts.\\

Further work is required to refine transition strategies, reduce performance costs, and ensure compatibility with Ethereum's evolving consensus model. Beyond Ethereum, the broader blockchain ecosystem will benefit from standardized benchmarks and cross-platform integration efforts for post-quantum security.\\

Finally, this thesis hopes to contribute to a growing body of work that aims to future-proof decentralized infrastructure, ensuring that core technologies like Ethereum remain secure, functional, and trustworthy in a context where quantum adversaries may soon become a reality.